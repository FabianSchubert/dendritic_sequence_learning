\documentclass[10pt,a4paper]{article}
\usepackage[utf8]{inputenc}
\usepackage[T1]{fontenc}
\usepackage{amsmath}
\usepackage{amsfonts}
\usepackage{amssymb}
\usepackage[left=1.5cm, right=1.5cm, top=2.50cm, bottom=2.50cm]{geometry}
\usepackage{multicol}
\author{Fabian Schubert}
\title{Nonlinear Dendritic Coincidence Detection for Supervised Learning}
\begin{document}
	\maketitle
	\begin{multicols}{2}
		\section{Introduction}
		
		In recent years, a growing body of research has addressed the 
		functional implications of the distinct physiology and anatomy of 
		cortical pyramidal neurons. In particular, on the theoretical side,
		we saw a paradigm shift from treating neurons as point-like electrical
		structures towards embracing the entire dendritic structure. This was 
		mostly due to the fact that experimental work uncovered dynamical properties
		of these cells that simply could not be accounted for by point models.
		
		An important finding was that the apical dendritic tree of
		cortical pyramidal neurons can act as a separate nonlinear synaptic 
		integration zone. Under certain conditions, a dendritic $\rm Ca^{2+}$ spike
		can be elicited that propagates towards the soma, causing rapid, bursting
		spiking activity. One of the cases in which dendritic spiking can occur
		was termed 'backpropagation-activated $\rm Ca^{2+}$ spike firing' 
		('BAC firing'): A single somatic spike can backpropagate towards the apical
		spike initiation zone, in turn significantly facilitating the initiation of 
		a dendritic spike. This reciprocal coupling is believed to act as a form of
		coincidence detection: If apical and basal synaptic input co-occurs, the 
		neuron can respond with a rapid burst of spiking activity. The firing rate
		of these temporal bursts exceeds the firing rate that is maximally achievable 
		under basal synaptic input alone, thus representing a form of temporal coincidence
		detection between apical and basal input.
		
		Naturally, these mechanisms also affect plasticity and thus learning
		within the cortex. While the interplay between basal and apical stimulation and
		its effect on synaptic efficacies is subject to ongoing research, there is
		some evidence that BAC-firing tends to shift plasticity towards long-term potentiation
		(LTP). Thus, coincidence between basal and apical input appears to also gate synaptic
		plasticity.
		
		In our work, we combined a phenomenological model predicting the output
		firing rate as a function of two streams of synaptic input (subsuming basal and apical input)
		with a BCM-like plasticity rule.
		
	\end{multicols}
	
\end{document}